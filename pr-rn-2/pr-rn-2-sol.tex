\input{Preamble.sty}

\begin{document}
%%%%%%%%%%%%%%%%%%%%%%%%%%%%%%%%%%Título%%%%%%%%%%%%%%%%%%%%%%%%%%%%%%%%%%%%%%
%%%%%%%%%%%%%%%%%%%%%%%%%%%%%%%%%%%%%%%%%%%%%%%%%%%%%%%%%%%%%%%%%%%%%%%%%%%%%%

\title{Práctica 2}
\author{Evelyn~G.~Coronel}

\affiliation{
Redes~Neuronales - Instituto~Balseiro\\}

\date[]{\lowercase{\today}} %%lw para lw, [] sin date

\begin{abstract}
Soluciones a los ejercicios de la práctica 2 de la materia de Redes neuronales. Esta práctica es acerca de la interacción entre neuronas.
\end{abstract} 
\maketitle
%%%%%%%%%%%%%%%%%%%%%%%%%%%%%%%%%%%%%%%%%%%%%%%%%%%%%%%%%%%%%%%%%%%%%%%%%%%%%%%%%%%
% Podemos usar cualquiera de los dos comandos: \input o \include para incluir el texto
%\input{./Capitulo1/cap1.tex}


\section{Ejercicio 1}

Las dos poblaciones de neuronas están descritas por las siguientes ecuaciones:
\begin{align}
    \tau \nicefrac{df_e}{dt} &= -f_e + g_{ee} f_e\Theta(f_e) - g_{ei}f_i\Theta(f_i) + I_e\\
    \tau \nicefrac{df_i}{dt} &= -f_i + g_{ie} f_e\Theta(f_e) - g_{ii}f_i\Theta(f_i) + I_i
\end{align}
donde $f_i$ y $f_e$ son las tasas de disparo, $g_{ee}$ y $g_{ii}$ son las conductancias asociadas a la autointeracción de la neuronas y los términos $g_{ie}$ y $g_{ei}$ son las conductancias de la interacción entre las neuronas.

Sabemos que la solución es estable cuando las derivadas se anulan
\begin{align}
    0= -f_e + g_{ee} f_e\Theta(f_e) - g_{ei}f_i\Theta(f_i) + I_e\\
    0= -f_i + g_{ie} f_e\Theta(f_e) - g_{ii}f_i\Theta(f_i) + I_i
\end{align}

Considerando que las tasas de disparo y las conductancias son positivas,  las condiciones de estabilidad quedan como
\begin{align}
     (1 -g_{ee}) f_e &= -g_{ei}f_i + I_e \label{fe}\\
    (1 + g_{ii}) f_i &=  g_{ie} f_e  + I_i
\end{align}

En cambio para la tasa $f_e$ dependiendo del valor de la conductancias puede tomar un valor negativo; por hipótesis asumimos que $f_e \ge 0$, esto implica que la actividad o la cantidad de spikes es nula.

Considerando la Ec.\,\ref{fe}, se obtiene una relación entre $f_e$ y $f_i$,
\begin{align}
	     f_e &=\frac{-g_{ei}f_i + I_e}{1 -g_{ee}}\\
	     0 &\le \frac{-g_{ei}f_i + I_e}{1 -g_{ee}}
\end{align}

Buscamos las condiciones para las cuales $f_e>=0$. Para esto se deben cumplir simultáneamente las ciertas condiciones en el denominador y numerador.

\begin{itemize}
	\item Si $1 > g_{ee}$ entonces  $I_e > g_{ei}f_i$.
	\item Si $1 < g_{ee}$ entonces  $I_e < g_{ei}f_i$.
\end{itemize}

Ya que simplificando el problema.
\subsection{Simulaciones}



\section{Ejercicio 2}


Implementamos la siguiente corriente

\begin{equation}
	I(t) = I_0 + I_{syn}(t)
\end{equation}



\end{document}