\input{Preamble.sty}
\usepackage{hyperref}
\begin{document}
%%%%%%%%%%%%%%%%%%%%%%%%%%%%%%%%%%Título%%%%%%%%%%%%%%%%%%%%%%%%%%%%%%%%%%%%%%
%%%%%%%%%%%%%%%%%%%%%%%%%%%%%%%%%%%%%%%%%%%%%%%%%%%%%%%%%%%%%%%%%%%%%%%%%%%%%%

\title{Práctica 3: Estadística de trenes de spikes }
\author{Evelyn~G.~Coronel}

\affiliation{
Redes~Neuronales - Instituto~Balseiro\\}

\date[]{\lowercase{\today}} %%lw para lw, [] sin date

\begin{abstract}
Soluciones a los ejercicios de la práctica 3 de la materia de Redes Neuronales. En esta práctica se estudia la estadísticas de los resultados obtenidos para un experimento. En mismo se excita una neurona con un estimulo oscilatorio.
\end{abstract} 
\maketitle
%%%%%%%%%%%%%%%%%%%%%%%%%%%%%%%%%%%%%%%%%%%%%%%%%%%%%%%%%%%%%%%%%%%%%%%%%%%%%%%%%%%

\section{Distribución de intervalos \texorpdfstring{$P(\tau)$}{}}


\end{document}

