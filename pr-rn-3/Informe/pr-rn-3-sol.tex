\input{Preamble.sty}
\usepackage{hyperref}
\usepackage{listings}
\usepackage{xcolor}


\definecolor{codegreen}{rgb}{0,0.6,0}
\definecolor{codegray}{rgb}{0.5,0.5,0.5}
\definecolor{codepurple}{rgb}{0.58,0,0.82}
\definecolor{backcolour}{rgb}{0.95,0.95,0.92}

%Code listing style named "mystyle"
\lstdefinestyle{mystyle}{
  backgroundcolor=\color{backcolour},   commentstyle=\color{codegreen},
  keywordstyle=\color{magenta},
  numberstyle=\tiny\color{codegray},
  stringstyle=\color{codepurple},
  basicstyle=\ttfamily\footnotesize,
  breakatwhitespace=false,         
  breaklines=true,                 
  captionpos=b,                    
  keepspaces=true,                 
  numbers=left,                    
  numbersep=5pt,                  
  showspaces=false,                
  showstringspaces=false,
  showtabs=false,                  
  tabsize=2
}
\lstset{style=mystyle}

\begin{document}
%%%%%%%%%%%%%%%%%%%%%%%%%%%%%%%%%%Título%%%%%%%%%%%%%%%%%%%%%%%%%%%%%%%%%%%%%%
%%%%%%%%%%%%%%%%%%%%%%%%%%%%%%%%%%%%%%%%%%%%%%%%%%%%%%%%%%%%%%%%%%%%%%%%%%%%%%

\title{Práctica 3: Estadística de trenes de spikes }
\author{Evelyn~G.~Coronel}

\affiliation{
Redes~Neuronales - Instituto~Balseiro\\}

\date[]{\lowercase{\today}} %%lw para lw, [] sin date

\begin{abstract}
Soluciones a los ejercicios de la práctica 3 de la materia de Redes Neuronales. En esta práctica se estudia la estadística de los resultados obtenidos para un experimento, donde se excita un receptor acústico de un saltamonte con un sonido.
\end{abstract} 
\maketitle
%%%%%%%%%%%%%%%%%%%%%%%%%%%%%%%%%%%%%%%%%%%%%%%%%%%%%%%%%%%%%%%%%%%%%%%%%%%%%%%%%%%
\section*{Introducción}



\subsection*{Estímulo}

\begin{figure}[H]
	\centering
	\includegraphics[width=0.5\textwidth]{../Graficos/stimulus.png}
	\caption{La figura superior representa la intensidad del estimulo en función de tiempo. La figura inferior representa todos los spikes observados en cada prueba.}
\end{figure}


La varianza del estimulo es de $\sigma^2_{s}= 31.649\,$dB$^2$
\section*{Distribución de intervalos \texorpdfstring{$P(\tau)$}{}}

\begin{figure}[H]
	\centering
	\includegraphics[width=0.5\textwidth]{../Graficos/p_isi.png}
	\caption{Distribución de probabilidad de los valores del ISI}
	\end{figure}


La distribución tiene una media de $\langle ISI \rangle = 84.68\,$ms y un varianza de $\sigma^2_{ISI}=3111.96\,$ms$^2$.
El factor $CV=0.659$


\section*{Distribución de número de spikes \texorpdfstring{$P(N)$}{}}

\begin{figure}[H]
	\centering
	\includegraphics[width=0.5\textwidth]{../Graficos/p_N.png}
	\caption{Distribución de probabilidad de la cantidad de spikes en cada realización del experimento.}
\end{figure}


La distribución tiene una media de $\langle N \rangle = 117.01$ y un varianza de $\sigma^2_{N}=183.195$.
El factor de Fano $F=1.567$


\section*{Histograma de la tasa de disparo \texorpdfstring{$r(t)$}{}}


\begin{figure}[H]
	\centering
	\includegraphics[width=0.5\textwidth]{../Graficos/r_t.png}
	\caption{Tasa de disparo subyacente}
\end{figure}

\section*{Filtro asociado a la neuronas \texorpdfstring{$D(\tau)$}{}}


Considerando que la señal es ruido blanco, es decir que $Q_{s,s}(\tau, \tau')= \sum_{spikes} \sigma^2_s \delta(\tau-\tau')$, se obtiene que el filtro asociado es 
\begin{align}
    D(\tau) &= \frac{Q_{r,s}(-\tau)}{\sigma^2}, \\
    \text{con } Q_{r,s}(-\tau) &= \sum_{spikes} S(t_{spike} - \tau)
\end{align}
donde $S(t)$ es el valor del estimulo al tiempo $t$

Para calcular este gráfico, lo que implemente fue el siguiente algoritmo.

\begin{lstlisting}[language=C++]
	int N_signal = 10001;

	for (int i = -500; i < 10000; ++i) //Tau
	{
		for (int j = 0; j < N_signal; ++j) //Signal
		{	t_evaluate= j+i;
			
			if (t_evaluate>=0 
					&& t_evaluate<N_signal 
					&& vector_spikes[j]!=0) 

					sum+=vector_signal[t_evaluate];
				
		}
		filtro_file<< i <<"\t"<< sum/sigma_s <<endl;
		sum=0.0;
	}
\end{lstlisting}


\begin{figure}[H]
	\centering
	\includegraphics[width=0.5\textwidth]{../Graficos/D_tau.png}
	\caption{Filtro lineal}
\end{figure}


\end{document}

