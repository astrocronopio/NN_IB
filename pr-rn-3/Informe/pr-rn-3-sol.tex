\input{Preamble.sty}
\usepackage{hyperref}
\begin{document}
%%%%%%%%%%%%%%%%%%%%%%%%%%%%%%%%%%Título%%%%%%%%%%%%%%%%%%%%%%%%%%%%%%%%%%%%%%
%%%%%%%%%%%%%%%%%%%%%%%%%%%%%%%%%%%%%%%%%%%%%%%%%%%%%%%%%%%%%%%%%%%%%%%%%%%%%%

\title{Práctica 3: Estadística de trenes de spikes }
\author{Evelyn~G.~Coronel}

\affiliation{
Redes~Neuronales - Instituto~Balseiro\\}

\date[]{\lowercase{\today}} %%lw para lw, [] sin date

\begin{abstract}
Soluciones a los ejercicios de la práctica 3 de la materia de Redes Neuronales. En esta práctica se estudia la estadística de los resultados obtenidos para un experimento. En mismo se excita una neurona con un estimulo oscilatorio.
\end{abstract} 
\maketitle
%%%%%%%%%%%%%%%%%%%%%%%%%%%%%%%%%%%%%%%%%%%%%%%%%%%%%%%%%%%%%%%%%%%%%%%%%%%%%%%%%%%

\section*{Estimulo}

\begin{figure}[H]
	\centering
	\includegraphics[width=0.5\textwidth]{../Graficos/stimulus.png}
	\caption{La intensidad del estimulo en función de tiempo. La misma representa un sonido que estimula a un insecto.}
\end{figure}


La varianza del estimulo es de $\sigma^2_{s}= 31.649\,$dB$^2$
\section*{Distribución de intervalos \texorpdfstring{$P(\tau)$}{}}

\begin{figure}[H]
	\centering
	\includegraphics[width=0.5\textwidth]{../Graficos/p_isi.png}
	\caption{Distribución de probabilidad de los valores del ISI}
	\end{figure}


La distribución tiene una media de $\langle ISI \rangle = 84.68\,$ms y un varianza de $\sigma^2_{ISI}=3111.96\,$ms$^2$.
El factor $CV=0.659$


\section*{Distribución de número de spikes \texorpdfstring{$P(N)$}{}}

\begin{figure}[H]
	\centering
	\includegraphics[width=0.5\textwidth]{../Graficos/p_N.png}
	\caption{Distribución de probabilidad de la cantidad de spikes en cada realización del experimento.}
\end{figure}


La distribución tiene una media de $\langle N \rangle = 117.01$ y un varianza de $\sigma^2_{N}=183.195$.
El factor de Fano $F=1.567$


\section*{Histograma de la tasa de disparo \texorpdfstring{$r(t)$}{}}


\begin{figure}[H]
	\centering
	\includegraphics[width=0.5\textwidth]{../Graficos/r_t.png}
	\caption{Tasa de disparo subyacente}
\end{figure}

\section*{Filtro asociado a la neuronas \texorpdfstring{$D(\tau)$}{}}


Considerando que la señal es ruido blanco, es decir que $Q_{s,s}(\tau, \tau')= \sum_{spikes} \sigma^2_s \delta(\tau-\tau')$, se obtiene que el filtro asociado es 
\begin{align}
    D(\tau) &= \frac{Q_{r,s}(-\tau)}{\sigma^2}, \\
    \text{con } Q_{r,s}(-\tau) &= \sum_{spikes} S(t_{spike} - \tau)
\end{align}
donde $S(t)$ es el valor del estimulo al tiempo $t$


\end{document}

